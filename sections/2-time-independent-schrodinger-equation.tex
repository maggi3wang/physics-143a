\section{2. Time-Independent Schr\"{o}dinger Equation} \hrule height 0.3pt \thinspace

\subsection{\underline{2.1 Stationary States}}
Suppose PE is independent of time, $V(\vec{r}, t) = V(\vec{r})$. \\
Separation of variables: $\Psi(\vec{r}, t) = \psi(\vec{r}) \varphi(t)$ \\

Eq of motion for $\varphi(t)$: $\varphi(t) = e^{-iEt/\hbar}$ \\

Eq of motion for $\psi(\vec{r})$ is the time-independent Schr\"{o}dinger equation:
$$-\frac{\hbar^2 }{2m} \dv[2]{\psi(\vec{r})}{x} + V(\vec{r}) \psi(\vec{r}) = E \psi(\vec{r}) $$ \\

TD of the wavefunction that corresponds to the constant $E$ is easily written once we solve the TISE: $\Psi_{E}(\vec{r}, t) = \psi_{E}(\vec{r}) e^{-iEt / \hbar}$ \\

\medskip

Properties of solutions for TI potentials: \\

\begin{itemize}[noitemsep,wide=0pt, leftmargin=\dimexpr\labelwidth + 2\labelsep\relax]
%\setlength{\itemindent}{-1.5em}
%\setlength{\parskip}{-1pt}
%\setlength{\parsep}{0pt}   
%\addtolength{\itemindent}{-2em}

    \item \textbf{The constant $E$ must be real.}
    
    \item \textbf{Stationary wavefunction.} \\
        $\mathcal{P}(\vec{r}, t) = |\psi_E(\vec{r}, t)|^2 = |\psi_E(\vec{r})|^2$ (TD cancels out).

    \item \textbf{Stationary wavefunction is a state of definite energy.} \\
        The total energy (kinetic plus potential) is the Hamiltonian: $H(x, p) = \frac{p^2}{2m} + V(x)$. \\

        Hamiltonian operator: $\widehat{H} = -\frac{\hbar^2}{2m} \pdv[2]{}{x} + V(x)$ \\
        Thus the TISE can be written as $\widehat{H} \psi = E \psi$ \\

        $\langle \widehat{H} \rangle = E$, $\langle \widehat{H} ^2 \rangle = E^2$, $\Delta E = \sqrt{\langle \widehat{H}^2 \rangle - \langle \widehat{H} \rangle ^2} = 0$

    \item Spatial part of stationary wavefunction can be chosen to be real. \\
        $\psi^*(\vec{r})$ is a soln w/ same $E$ \\
        Solns can be chosen to be real: $\psi(\vec{r}) + \psi^*(\vec{r})$ and $\frac{\psi(\vec{r}) - \psi^*(\vec{r})}{i}$.

    \item \textbf{Parity symmetry: even and odd wavefunctions.}
        Suppose $V(-\vec{r}) = V(\vec{r})$. Then, $\psi_E(-\vec{r})$ is a soln w the same energy. \\
        $\psi_E(\vec{r}) + \psi_E(-\vec{r})$ is even under reflection, $\psi_E(\vec{r}) - \psi_E(-\vec{r})$ is odd. \\
        When the potential is symmetric under reflection, we can choose the stationary states to be either even or odd under reflection.

    \item \textbf{Orthogonality/orthonormality.} \\
        $\int \psi_m (\vec{r})^* \psi_n (\vec{r}) d^3 \vec{r} = \delta_{mn}$ where $\delta_{mn}$ is 0 if $m \neq n$ and 1 if $m = n$.

    \item \textbf{Linearity.} \\
        The SE is linear. Given stationary states, a linear combo of these
            $$\psi(\vec{r}, t) = \sum_{n} c_n \psi_n(\vec{r}, t) = \sum_n c_n \psi_n(\vec{r}) e^{-\frac{i}{\hbar} E_n t}$$
        where $c_n$ are complex constants, is a solution to the TDSE 
        $$i \hbar \pdv{\psi(\vec{r}, t)}{t} = \widehat{H} \psi(\vec{r}, t)$$

    \item \textbf{Time evolution.}
        Given $$\psi(\vec{r}, 0) = \sum_n c_n \psi_n(\vec{r}, 0) = \sum_n c_n \psi_n (\vec{r})$$
        at time $t$, the time evolution is 
        $$\psi(\vec{r}, t) = \sum_n c_n \psi_n(\vec{r}, t) = \sum_n c_n \psi_n(\vec{r}) e^{-\frac{i}{\hbar} E_n t}$$

        Once we've expanded a given initial wavefunction in terms of a linear combo of the stationary wavefunctions $\psi_n(\vec{r})$, the time evolution follows simply by putting a factor of $e^{-i/\hbar E_n t}$ to each term containing $\psi_n(\vec{r})$.

    \item \textbf{Normalization.} \\
        The constant coefficients are constrained by $\sum_n |c_n|^2 = 1$

    \item \textbf{Completeness.} \\
        The stationary states form a complete set if
            $$\sum_n \psi_n (\vec{r'}, t)^* \psi_{n} (\vec{r}, t) = \delta^3 (\vec{r'} - \vec{r})$$

        where $\delta^3(\vec{r'} - \vec{r})$ is the Dirac-delta function in 3D defined by $$\int d^3 \vec{r'} \psi(\vec{r'}, t) \delta^3(\vec{r'} - \vec{r}) = \psi(\vec{r}, t)$$
\end{itemize}

\textbf{Euler's formula}: $e^{i \theta} = \cos \theta + i \sin \theta$

\textbf{Delta function}: Given $f(x)$, $\delta(x - x')$ is defined as $f(x') = \int f(x) \delta(x - x') dx$ \\
$\int \delta(x - x') dx = 1$, note this is not the area \\

    $\delta_{\alpha}(x) = \frac{1}{\alpha \sqrt{\pi}} e^{-\frac{x^2}{\alpha^2}}$,
    $\delta_{\alpha}(x) = \frac{1}{\pi x} \sin(\frac{x}{a})$, 
    $\delta_{\alpha}(x) = \frac{\alpha}{\pi x^2} \sin^2 (\frac{x}{\alpha})$ \\

\subsection{\underline{One-dimensional systems}}
Wavefunction for a system containing a single particle of mass $m$ in 1D with TI potentials.
    $$- \frac{\hbar^2}{2m} \dv[2]{\psi(x)}{x} + V(x) \psi(x) = E \psi(x)$$

Once we find the wavefunction $\psi_E(x)$ of energy $E$, its time dependence follows easily:
    $$\psi_E(x, t) = \psi_E(x) e^{-\frac{i}{\hbar} E t}$$

\textbf{Boundary conditions} \\
1. When the potential $V(x)$ has a finite jump at $x = a$, both $\psi(x)$ and $\psi'(x)$ are continuous across $x = a$.
2. When the potential $V(x)$ has an infinite jump at $x = a$, $\psi(x)$ is continuous but $\psi'(x)$ is discontinuous across $x = a$.

Futhermore, the wavefunction must vanish at $x = \pm \infty$ for a normalizable wavefunction.

\subsection{\underline{2.2 The Infinite Square Well}}
Suppose
    $$V(x) = \begin{cases} 0, \textrm{if } 0 \leq x \leq a \\ \infty, \textrm{otherwise} \end{cases}$$

    $$\psi(x) = 0 \textrm{ for } x < 0 \textrm{ and } x > a$$
For $0 \leq x \leq a$, we have $V(x) = 0$ and the Schr\"odinger equation reduces to 
    $$\psi''(x) + k^2 \psi(x) = 0 \textrm{, where } k = \sqrt{\frac{2mE}{\hbar^2}} \textrm{ and } E > 0$$

Classic simple harmonic oscillator, $\psi(x) = A \sin(kx) + B \cos(kx)$ \\

Boundary conditions \\

$$E_n = \frac{\hbar^2 k_n^2}{2m} - \frac{n^2 \pi^2 \hbar^2}{2ma^2}$$

$$\psi_n(x) = \sqrt{\frac{2}{a}} \sin(\frac{n \pi}{a} x)$$

$\psi_1$ is the ground state, others are excited states. \\

Properties of $\psi_n(x)$: \\
1. Alternatively even and odd. \\
2. As you go up in energy, each successive state has one more node. \\
3. They are mutually orthogonal, in the sense that $\int \psi_m(x)* \psi_n(x) dx = 0$ whenever $m \neq n$. \\

$\int \psi_m (x)* \psi_n(x) dx = \delta_{mn}$
where $\delta_{mn}$ (Kronecker delta) is 0 if $m \neq n$ and 1 if $m=n$. We say that the $\phi$'s are orthonormal.

4. They are complete, in the sense that any other function, $f(x)$, can be expressed as a linear combination of them (Fourier series), Dirichlet's theorem:
    $$f(x) = \sum_{n=1}^{\infty} c_n \psi_n(x) = \sqrt{\frac{2}{a}} \sum_{n=1}^{\infty} c_n \sin(\frac{n \pi}{a} x)$$

Fourier's trick: $c_n = \int \psi_n(x)^* f(x) dx$ \\
$$c_m = \frac{2}{a} \int_0^a f(x) \sin(\frac{m \pi x}{a}) dx$$

$|c_n|^2$ tells you the probability that a measurement of the energy would yield the value $E_n$. \\

Sum of these probabilities should be 1: 
    $$\sum_{n=1}^{\infty} |c_n|^2 = 1$$

The expectation value of the energy is
    $$\langle H \rangle = \sum_{n=1}^{\infty} |c_n|^2 E_n$$

Conservation of energy in QM

\subsection{\underline{2.3 The Harmonic Oscillator}}
Hooke's law: $F = -kx = m \frac{d^2 x}{d t^2}$ \\
Solution is $x(t) = A \sin(\omega t) + B \cos(\omega t)$, where $\omega = \sqrt{\frac{k}{m}}$, $V(x) = \frac{1}{2} k x^2$. \\

Taylor series: $V(x) = V(x_0) + V'(x_0) (x - x_0) + \frac{1}{2} V''(x_0)(x- x_0)^2 + \cdots$ \\

The Schr\"{o}diner Equation for the harmonic oscillator: $-\frac{\hbar^2}{2m} \dv[2]{\psi}{x} + \frac{1}{2} m \omega^2 x^2 \psi = E \psi$ \\
Introduce $\xi \equiv \sqrt{\frac{m \omega}{\hbar}} x$, so we have $\dv[2]{\psi}{\xi} = (\xi^2 - K) \psi$, where $K \equiv \frac{2E}{\hbar \omega}$. \\

The recursion formula: $a_{j+2} = \frac{(2j + 1 - K)}{(j + 1)(j + 2)} a_j$ \\
The complete solution is $h(\xi) = h_{\text{even}}(\xi) + h_{\text{odd}}(\xi)$ \\

$K = 2n + 1$, so $E_n = (n + \frac{1}{2}) \hbar \omega$ \\

Recursion formula for allowed $K$: $a_{j+2} = \frac{-2(n - j)}{(j+1)(j+2)} a_j$ \\

Hermite polynomials: $H_0 = 1$, $H_1 = 2 \xi$, $H_2 = 4 \xi^2 - 2$, $H_3 = 8 \xi^3 - 12 \xi$, $H_4 = 16 \xi^4 - 48 \xi^2 + 12$, $H_5 = 32 \xi^5 - 160 \xi^3 + 120 \xi$ \\

The normalized stationary states: $\psi_n(x) = (\frac{m \omega}{\pi \hbar})^{1/4} \frac{1}{\sqrt{2^n n!}} H_n(\xi) e^{-\xi^2 / 2}$ \\

Rodrigues formula: $H_n(\xi) = (-1)^n e^(\xi^2) (\dv{}{\xi})^n e^{-\xi^2}$

\subsection{\underline{2.4 The Free Particle}}
$$\pdv[2]{\xi}{x} = -k^2 \xi, k = \frac{\sqrt{2mE}}{\hbar}$$

General solution to the TISE: wave packet, $\Psi(x, t) = \frac{1}{\sqrt{2\pi}} \int_{-\infty}{\infty} \psi(k) e^{i (kx - \frac{\hbar k^2}{2m} t)} dk$ \\
$$\phi(x) = \frac{1}{\sqrt{2 \pi}} \int_{-\infty}^{+\infty} \Psi(x, 0) e^{-ikx} dx$$

Plancherel's theorem: $$f(x) = \frac{1}{\sqrt{2 \pi}} \int_{-\infty}^{\infty} F(k) e^{ikx} dk \leftrightarrow F(k) = \frac{1}{\sqrt{2\pi}} \int_{-\infty}^{\infty} f(x) e^{-kx} dx$$

$F(k)$ is the Fourier transform of $f(x)$; $f(x)$ is the inverse Fourier transform of $F(k)$ \\

Phase velocity: speed of individual ripples; grouop velocity: speed of the envelope \\

Dispersion relation: the formula for $\omega$ as a function of $k$

\subsection{\underline{2.5 The Delta-Function Potential}}

\subsection{\underline{2.6 The Finite Square Well}}


