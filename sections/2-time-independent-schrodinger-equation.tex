\section{2. Time-Independent Schr\"{o}dinger Equation} \hrule height 0.3pt \thinspace

\subsection{\underline{2.1 Stationary States}}
Suppose PE is independent of time, $V(\vec{r}, t) = V(\vec{r})$. \\
Separation of variables: $\Psi(\vec{r}, t) = \psi(\vec{r}) \varphi(t)$ \\

Eq of motion for $\varphi(t)$: $\varphi(t) = e^{-iEt/\hbar}$ \\

Eq of motion for $\psi(\vec{r})$ is the time-independent Schr\"{o}dinger equation:
$$-\frac{\hbar^2 }{2m} \dv[2]{\psi(\vec{r})}{x} + V(\vec{r}) \psi(\vec{r}) = E \psi(\vec{r}) $$ \\

TD of the wavefunction that corresponds to the constant $E$ is easily written once we solve the TISE: $\Psi_{E}(\vec{r}, t) = \psi_{E}(\vec{r}) e^{-iEt / \hbar}$ \\

\medskip

Why separable solutions? \\

\smallskip

1. Stationary states - time-dependence cancels out
    $$|\Psi(x,t)|^2 = \Psi^* \Psi = \psi^* e^{+iEt/\hbar} \psi e^{-iEt/\hbar} = |\Psi(x)|^2$$

    Same thing happens in calculating the expectation value of any dynamical variable. Every expectation value is constant in time. \\

\smallskip

2. There are states of definite total energy. The total energy (kinetic plus potential) is the Hamiltonian: $H(x, p) = \frac{p^2}{2m} + V(x)$. \\

Hamiltonian operator: $\widehat{H} = -\frac{\hbar^2}{2m} \pdv[2]{}{x} + V(x)$ \\
Thus the TISE can be written as $\widehat{H} \psi = E \psi$ \\
Variance of $H$: $\sigma_{H}^2 = \langle H^2 \rangle - \langle H \rangle ^2 = E^2 - E^2 = 0$ \\
A separable solution has the property that every measurement of the total energy is certain to return the value $E$. \\

\smallskip

3. The general solution is a linear combination of separable solutions. There is a different wave function for each allowed energy: $\Psi_1(x,t) = \psi_1(x) e^{-iE_1 t / \hbar}, \Psi_2(x,t) = \psi_2 (x) e^{-iE_2 t \hbar}, ...$

Now the time dependent Schr\"{o}dinger equation has the property that any linear combo of solutions is itself a solution.

$$\Psi(x, t) = \sum_{n=1}^{\infty} c_n \psi_n(x) e^{-iE_n t / \hbar} = \sum_{n=1}^{\infty} c_n \Psi_n(x, t)$$

\smallskip

Euler's formula: $e^{i \theta} = \cos \theta + i \sin \theta$

\subsection{\underline{2.2 The Infinite Square Well}}

Suppose
    $$V(x) = \begin{cases} 0, \textrm{if } 0 \leq x \leq a \\ \infty, \textrm{otherwise} \end{cases}$$

Classic simple harmonic oscillator, $\psi(x) = A \sin(kx) + B \cos(kx)$ \\

Boundary conditions \\

$$E_n = \frac{\hbar^2 k_n^2}{2m} - \frac{n^2 \pi^2 \hbar^2}{2ma^2}$$

$$\psi_n(x) = \sqrt{\frac{2}{a}} \sin(\frac{n \pi}{a} x)$$

$\psi_1$ is the ground state, others are excited states. \\

Properties of $\psi_n(x)$: \\
1. Alternatively even and odd. \\
2. As you go up in energy, each successive state has one more node. \\
3. They are mutually orthogonal, in the sense that $\int \psi_m(x)* \psi_n(x) dx = 0$ whenever $m \neq n$. \\

$\int \psi_m (x)* \psi_n(x) dx = \delta_{mn}$
where $\delta_{mn}$ (Kronecker delta) is 0 if $m \neq n$ and 1 if $m=n$. We say that the $\phi$'s are orthonormal.

4. They are complete, in the sense that any other function, $f(x)$, can be expressed as a linear combination of them (Fourier series), Dirichlet's theorem:
    $$f(x) = \sum_{n=1}^{\infty} c_n \psi_n(x) = \sqrt{\frac{2}{a}} \sum_{n=1}^{\infty} c_n \sin(\frac{n \pi}{a} x)$$

Fourier's trick: $c_n = \int \psi_n(x)^* f(x) dx$ \\
$$c_m = \frac{2}{a} \int_0^a f(x) \sin(\frac{m \pi x}{a}) dx$$

$|c_n|^2$ tells you the probability that a measurement of the energy would yield the value $E_n$. \\

Sum of these probabilities should be 1: 
    $$\sum_{n=1}^{\infty} |c_n|^2 = 1$$

The expectation value of the energy is
    $$\langle H \rangle = \sum_{n=1}^{\infty} |c_n|^2 E_n$$

Conservation of energy in QM

\subsection{\underline{2.3 The Harmonic Oscillator}}
Hooke's law: $F = -kx = m \frac{d^2 x}{d t^2}$ \\
Solution is $x(t) = A \sin(\omega t) + B \cos(\omega t)$, where $\omega = \sqrt{\frac{k}{m}}$, $V(x) = \frac{1}{2} k x^2$. \\

Taylor series: $V(x) = V(x_0) + V'(x_0) (x - x_0) + \frac{1}{2} V''(x_0)(x- x_0)^2 + \cdots$ \\

The Schr\"{o}diner Equation for the harmonic oscillator: $-\frac{\hbar^2}{2m} \dv[2]{\psi}{x} + \frac{1}{2} m \omega^2 x^2 \psi = E \psi$ \\
Introduce $\xi \equiv \sqrt{\frac{m \omega}{\hbar}} x$, so we have $\dv[2]{\psi}{\xi} = (\xi^2 - K) \psi$, where $K \equiv \frac{2E}{\hbar \omega}$. \\

The recursion formula: $a_{j+2} = \frac{(2j + 1 - K)}{(j + 1)(j + 2)} a_j$ \\
The complete solution is $h(\xi) = h_{\text{even}}(\xi) + h_{\text{odd}}(\xi)$ \\

$K = 2n + 1$, so $E_n = (n + \frac{1}{2}) \hbar \omega$ \\

Recursion formula for allowed $K$: $a_{j+2} = \frac{-2(n - j)}{(j+1)(j+2)} a_j$ \\

Hermite polynomials: $H_0 = 1$, $H_1 = 2 \xi$, $H_2 = 4 \xi^2 - 2$, $H_3 = 8 \xi^3 - 12 \xi$, $H_4 = 16 \xi^4 - 48 \xi^2 + 12$, $H_5 = 32 \xi^5 - 160 \xi^3 + 120 \xi$ \\

The normalized stationary states: $\psi_n(x) = (\frac{m \omega}{\pi \hbar})^{1/4} \frac{1}{\sqrt{2^n n!}} H_n(\xi) e^{-\xi^2 / 2}$ \\

Rodrigues formula: $H_n(\xi) = (-1)^n e^(\xi^2) (\dv{}{\xi})^n e^{-\xi^2}$

\subsection{\underline{2.4 The Free Particle}}
$$\pdv[2]{\xi}{x} = -k^2 \xi, k = \frac{\sqrt{2mE}}{\hbar}$$

General solution to the TISE: wave packet, $\Psi(x, t) = \frac{1}{\sqrt{2\pi}} \int_{-\infty}{\infty} \psi(k) e^{i (kx - \frac{\hbar k^2}{2m} t)} dk$ \\
$$\phi(x) = \frac{1}{\sqrt{2 \pi}} \int_{-\infty}^{+\infty} \Psi(x, 0) e^{-ikx} dx$$

Plancherel's theorem: $$f(x) = \frac{1}{\sqrt{2 \pi}} \int_{-\infty}^{\infty} F(k) e^{ikx} dk \leftrightarrow F(k) = \frac{1}{\sqrt{2\pi}} \int_{-\infty}^{\infty} f(x) e^{-kx} dx$$

$F(k)$ is the Fourier transform of $f(x)$; $f(x)$ is the inverse Fourier transform of $F(k)$ \\

Phase velocity: speed of individual ripples; grouop velocity: speed of the envelope \\

Dispersion relation: the formula for $\omega$ as a function of $k$

\subsection{\underline{2.5 The Delta-Function Potential}}

\subsection{\underline{2.6 The Finite Square Well}}


