\section{2. Time-Independent Schr\"{o}dinger Equation} \hrule height 0.3pt \thinspace

\subsection{\underline{2.1 Stationary States}}
Separation of variables: $\Psi(x, t) = \psi(x) \varphi(t)$ \\

$\varphi(t) = e^{-iEt/\hbar}$, separable solutions: $\Psi(x, t) = \psi(x) e^{-iEt / \hbar}$ \\

Time-independent Schr\"{o}dinger equation:
$$-\frac{\hbar^2 }{2m} \dv[2]{\psi}{x} + V \psi = E \psi $$ \\

Why separable solutions? \\
1. Stationary states - time-dependence cancels out
    $$|\Psi(x,t)|^2 = \Psi^* \Psi = \psi^* e^{+iEt/\hbar} \psi e^{-iEt/\hbar} = |\Psi(x)|^2$$

    Same thing happens in calculating the expectation value of any dynamical variable. Every expectation value is constant in time. \\

2. There are states of definite total energy. The total energy (kinetic plus potential) is the Hamiltonian: $H(x, p) = \frac{p^2}{2m} + V(x)$. \\

Hamiltonian operator: $\widehat{H} = -\frac{\hbar^2}{2m} \pdv[2]{}{x} + V(x)$ \\
Variance of $H$: $\sigma_{H}^2 = \langle H^2 \rangle - \langle H \rangle ^2 = E^2 - E^2 = 0$ \\
A separable solution has the property that every measurement of the total energy is certain to return the value $E$. \\

3. The general solution is a linear combination of separable solutions. There is a different wave function for each allowed energy: $\Psi_1(x,t) = \psi_1(x) e^{-iE_1 t / \hbar}, \Psi_2(x,t) = \psi_2 (x) e^{-iE_2 t \hbar}, ...$

Now the time dependent Schr\"{o}dinger equation has the property that any linear combo of solutions is itself a solution.

$$\Psi(x, t) = \sum_{n=1}^{\infty} c_n \psi_n(x) e^{-iE_n t / \hbar} = \sum_{n=1}^{\infty} c_n \Psi_n(x, t)$$

Euler's formula: $e^{i \theta} = \cos \theta + i \sin \theta$

\subsection{\underline{2.2 The Infinite Square Well}}

Suppose
    $$V(x) = \begin{cases} 0, \textrm{if } 0 \leq x \leq a \\ \infty, \textrm{otherwise} \end{cases}$$

Classic simple harmonic oscillator, $\psi(x) = A \sin(kx) + B \cos(kx)$ \\

Boundary conditions \\

$$E_n = \frac{\hbar^2 k_n^2}{2m} - \frac{n^2 \pi^2 \hbar^2}{2ma^2}$$

$$\phi_n(x) = \sqrt(\frac{2}{a} \sin(\frac{n \pi}{a} x)$$

$\phi_1$ is the ground state, others are excited states. \\

Properties of $\phi_n(x)$: \\
1. Alternatively even and odd. \\
2. As you go up in energy, each successive state has one more node. \\
3. They are mutually orthogonal, in the sense that $\int \psi_m(x)* \psi_n(x) dx = 0$ whenever $m \neq n$. \\

$\int \psi_m (x)* \psi_n(x) dx = \delta_{mn}$
where $\delta_{mn}$ (Kronecker delta) is 0 if $m \neq n$ and 1 if $m=n$. We say that the $\phi$'s are orthonormal.

4. They are complete, in the sense that any other function, $f(x)$, can be expressed as a linear combination of them (Fourier series), Dirichlet's theorem:
    $$f(x) = \sum_{n=1}^{\infty} c_n \psi_n(x) = \sqrt{\frac{2}{a}} \sum_{n=1}^{\infty} c_n \sin(\frac{n \pi}{a} x)$$

Fourier's trick: $c_n = \int \psi_n(x)^* f(x) dx$ \\

$|c_n|^2$ tells you the probability that a measurement of the energy would yield the value $E_n$. \\

Sum of these probabilities should be 1: 
    $$\sum_{n=1}^{\infty} |c_n|^2 = 1$$

The expectation value of the energy is
    $$\langle H \rangle = \sum_{n=1}^{\infty} |c_n|^2 E_n$$

Conservation of energy in QM

\subsection{\underline{2.3 The Harmonic Oscillator}}
Hooke's law: $F = -kx = m \frac{d^2 x}{d t^2}$

\subsection{\underline{2.4 The Free Particle}}

\subsection{\underline{2.5 The Delta-Function Potential}}

\subsection{\underline{2.6 The Finite Square Well}}


