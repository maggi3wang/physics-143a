\scriptsize

\section{3. Principles of QM} \hrule height 0.3pt \thinspace

\subsection{\underline{Axiomatic principles}}

\textbf{State vector axiom:} State vector at $t$ is ket $\psi(t)$, or $|\psi \rangle$, bra state.

\textbf{Probability axiom:} Given a system in state $|\psi \rangle$, a measurement will find it in state $|\phi \rangle$ with probability amplitude $\langle \phi | \psi \rangle$. 

\textbf{Hermitian operator axiom:} Physical observable is represented by a linear and Hermitian operator.

\textbf{Measurement axiom:} Measurement of a physical observable results in eigenvalue of observable. Observable $\widehat{A}$, we have $\widehat{A} | a \rangle = a | a \rangle$, where $a$ is eigenvalue and $|a \rangle$ is eigenvector. Measurement of the physical quantity represented by $\widehat{A}$ collapses the state $|\psi \rangle$ before measurement into an eigenstate $|a \rangle$ of $\widehat{A}$.

\textbf{Time evolution axiom:} $i \hbar \pdv{}{t} | \psi(t) \rangle = \widehat{H} |\psi(t) \rangle$, w/o consider $x$ or $p$.

\subsection{\underline{Vector space}}
State vector is neither in position nor momentum space. \\
Basis vectors:
$|0 \rangle = \begin{bmatrix}
    1 \\
    0 \\
    : \\
    0
    \end{bmatrix}$,
$|1 \rangle = \begin{bmatrix}
    0 \\
    1 \\
    : \\
    0
    \end{bmatrix}$,
$|n \rangle = \begin{bmatrix}
    0 \\
    0 \\
    : \\
    1
\end{bmatrix}$ (in $n$th pos).

    \textbf{Linearity}: Because the SE is linear, given two states $|\psi_1(t) \rangle$ and $|\psi_2(t) \rangle$, $|\psi(t) \rangle = c_1 |\psi_1(t) \rangle + c_2|\psi_2(t)\rangle$ is also a sol. ($c$'s are complex).

\textbf{Properties of a vector space}
% This is pretty self-explanatory

\textbf{Dual vector space}
$c|\psi \rangle$ is mapped to $c* \langle \psi |$. Given a vector, $|\psi \rangle = \begin{bmatrix} : \\ \alpha \\ : \end{bmatrix}$, the dual vector is $\langle \psi | = \begin{bmatrix} \cdots & \alpha^* & \cdots \end{bmatrix}$.

Dual basis vectors are $\langle 0 | = \begin{bmatrix} 1 & 0 & \cdots \end{bmatrix}$, $\cdots$, $\langle n| \begin{bmatrix} 0 & \cdots & 1 \end{bmatrix}$.

\textbf{Inner product}: $\langle \phi | \psi \rangle = c$, where $c$ is complex.

$\langle \phi | \psi \rangle = \langle \psi | \phi \rangle^*$ $\rightarrow$ $\langle \psi | \psi \rangle$ is real, positive, and finite for a normalizable ket vector. Can choose $\langle \psi | \psi \rangle = 1$. $\langle \psi_m | \psi_n \rangle = \delta_{mn}$

\subsection{\underline{Operators}}
A matrix operator $\widehat{A}$ acting on a state vector $|\psi \rangle$ transforms it into another state vector $|\phi \rangle$, $\widehat{A} |\psi \rangle = | \phi \rangle$. It is linear.

\textbf{Properties of operators}
% Also pretty self-explanatory.

\textbf{Hermitian conjugate (Hermitian adjoint) operator in the dual space} \\
Hermitian adjoint operator $\widehat{A}^{\dag}$ acts on the dual vector $\langle \psi |$ from the right as $\langle \psi | \widehat{A} ^{\dag} \rangle$, where $\widehat{A}^{\dag} = (\widehat{A})^{T*}$.
$$(\widehat{A} | \psi \rangle)^{\dag} = |\psi \rangle^{\dag} \widehat{A}^{\dag} = \langle \psi | \widehat{A}^{\dag} \qquad \langle \psi | = | \psi \rangle^{\dag} \qquad \langle \psi | ^{\dag} = | \psi \rangle$$
$$(\widehat{A}\widehat{B})^{\dag} = (\widehat{A} \widehat{B})^{T*} = (\widehat{B}^T \widehat{A}^T)^* = \widehat{B}^{T*} \widehat{A}^{T*} = \widehat{B}^{\dag} \widehat{A}^{\dag}, \quad (c\widehat{A})^{\dag} = c^* \widehat{A}^{\dag}$$

\textbf{Outer product operators}: $| \psi \rangle \langle \phi | \qquad [|\psi \rangle \langle \phi |] \chi \rangle  = | \psi \rangle \langle \phi | \chi \rangle$

\textbf{Matrix elements of operators}
$$\langle \phi | \widehat{A} | \psi \rangle \textrm{ (complex num)}$$

Hermitian equiv to complex conj $\langle \phi | \widehat{A} | \psi \rangle^{\dag} = \langle \psi | \widehat{A}^{\dag} | \phi \rangle = \langle \phi | \widehat{A} | \psi \rangle ^*$

\textbf{Hermitian operators}: $\widehat{A}^{\dag} = \widehat{A}$, so given $\widehat{A} | \phi \rangle$ in the vector space, we have $\langle \psi | \widehat{A}^{\dag} = \langle \phi | \widehat{A}$ in the dual vector space.

\textbf{Matrix elements of a Hermitian operator}
$$\langle \phi | \widehat{A} | \psi \rangle^{\dag} = \langle \phi | \widehat{A} | \psi \rangle^* = \langle \psi | \widehat{A}^{\dag} | \phi \rangle = \langle \psi | \widehat{A} | \phi \rangle$$

Hermitian operator, real expectation vals: $\langle \psi | \widehat{A} | \phi \rangle ^* = \langle \psi | \widehat{A} | \phi \rangle \equiv \langle \widehat{A} \rangle$

Same result whether $\widehat{A}$ acts to right or left: $\langle \phi | \widehat{A} | \psi \rangle = \langle \phi | \widehat{A}^{\dag} | \psi \rangle$

\textbf{Eigenvals and eigenvecs of Hermitian operators}: $\widehat{A} |a_n \rangle = a_n | a_n \rangle$

Normalized eigvecs $\langle a_m | a_n \rangle = \delta_{mn}$. Gram-Schmidt, degenerate evec.

\textbf{Completeness of eigenvector of a Hermitian operator}
Set $|a_n \rangle$ is complete if $\sum_n |\langle a_n | \psi \rangle|^2 = 1$.
$\sum_n |a_n \rangle \langle a_n | = 1$ (identity operator)

\textbf{Continuous spectra of a Hermitian operator}

Hermitian operator $\widehat{A}$, $\widehat{A} | a \rangle = a | a \rangle$, where $a$ is continuous. 

$\int da' \langle a' | \widehat{A} | a \rangle = a \int da' \langle a' | a \rangle = \int da' a' \langle a' | a \rangle \rightarrow \langle a' | a \rangle = \delta (a' - a)$

Continuous condition: $\int da | a \rangle \langle a | = 1$

\textbf{Gram-Schmidt orthogonalization procedure}

Eigval (like energy level) is $n$-fold degenerate: $n$ states w same eigval.

Orthogonal eigenstates $\rightarrow$ no degeneracy.

1. Normalize each state and define $\alpha_i = \frac{\alpha_i}{\sqrt{\langle a_i | a_i \rangle}}$. 2. $|\alpha'_1 \rangle = | \alpha_1 \rangle$.

3. $|\alpha'_2 \rangle = \frac{|\alpha_2 \rangle - |\alpha_1 \rangle \langle \alpha_1 | \alpha_2 \rangle}{\sqrt{\langle \alpha_2 | \alpha_2 \rangle - \langle \alpha_1 | \alpha_2 \rangle \langle \alpha_2 | \alpha_1 \rangle}} = \frac{|\alpha_2 \rangle - |\alpha_1 \rangle \langle \alpha_1 | \alpha_2 \rangle}{\sqrt{1 - \langle \alpha_1 | \alpha_2 \rangle \langle \alpha_2 | \alpha_1 \rangle}}$

4. Subtract components of $|\alpha_3 \rangle$ along $|\alpha_1 \rangle$ and $|\alpha_2 \rangle$, $|\alpha_3 \rangle - |\alpha_1 \rangle \langle \alpha_1 | \alpha_3 \rangle - | \alpha_2 \rangle \langle \alpha_2 | \alpha_3 \rangle$, normalize and promote to $|\alpha'_3 \rangle$.
...

\subsection{\underline{Position and momentum representation}}

$\widehat{\vec{r}} | \vec{r} \rangle = \vec{r} | \vec{r} \rangle \quad \langle \vec{r'} | \vec{r} \rangle = \delta^3 (\vec{r'} - \vec{r}), \int d^3 \vec{r} |\vec{r} \rangle \langle \vec{r} | = 1, \langle \vec{r'} | \hat{\vec{r}} | \vec{r} \rangle = \vec{r} \delta^3(\vec{r'} - \vec{r})$

$\widehat{\vec{p}} | \vec{p} \rangle = \vec{p} | \vec{p} \rangle \quad \langle \vec{p'} | \vec{p} \rangle = \delta^3(\vec{p'} - \vec{p}), \int d^3 \vec{p} | \vec{p} \rangle \langle \vec{p} | = 1$

State vector $| \psi(t) \rangle$ in position space (scalar): $\langle \vec{r} | \psi(x, t) \rangle \equiv \psi(\vec{r}, t)$

$\langle \psi | \widehat{\vec{p}} | \psi \rangle = \dv{}{t} \langle \psi | \widehat{\vec{r}} | \psi \rangle m$

Representation of momentum operator in position space: $\widehat{\vec{p}} = -i \hbar \vec{\grad}$.

$\langle x | \widehat{p} | x' \rangle = -i \hbar \pdv{}{x} \delta(x - x') = -i \hbar \pdv{}{x} \langle x | x' \rangle$.

$\widehat{p} = -i \hbar \pdv{}{x}$ is Hermitian, $\pdv{}{x}$ is not.

$$\langle x | \widehat{p} | p \rangle = p \langle x | p \rangle = -i \hbar \pdv{}{x} \langle x | p \rangle$$

The solution is $\langle x | p \rangle = \frac{1}{\sqrt{2 \pi \hbar}} e^{\frac{i}{\hbar} px}$. 

In 3D, $\langle \vec{r} | \vec{p} \rangle = \frac{1}{(2 \pi \hbar)^{3/2}} e^{\frac{i}{\hbar} \vec{p} \vec{r}}$.

We can write the normalized wavefunction of definite position in momentum space, $\langle p | x \rangle = \langle x | p \rangle^*$.

$\widehat{\vec{r}}$ is Hermitian and $\langle \phi | \widehat{\vec{r}}^{\dag} | \psi \rangle = \langle \phi | \widehat{\vec{r}} | \psi \rangle$.

\textbf{Operators and wavefunction in position representation}

Position and momentum operators in pos space: $\widehat{\vec{r}} = \vec{r}$, $\widehat{\vec{p}} = -i \hbar \vec{\grad}$.

$\widehat{O}(\widehat{\vec{r}}, \widehat{\vec{p}}) = \widehat{O} (\vec{r}, -i \hbar \vec{\grad})$

The expectation val of the observable should be indep of representation. In state $\psi(t)$, $\langle \widehat{O} \rangle = \langle \psi(t) | \widehat{O} | \psi(t) \rangle$.

Insert $\int d^2 \vec{r} | \vec{r} \rangle \langle \vec{r} | = 1$ to get $\langle \widehat{O} \rangle = \int d^2 \vec{r} \langle \psi(t) | \vec{r} \rangle \langle \vec{r} | \widehat{O} | \psi(t) \rangle$

$\psi(\vec{r}, t) = \langle \vec{r} | \psi(t) \rangle, \qquad \psi(\vec{r}, t)^* = \langle \vec{r} | \psi(t) \rangle^* = \langle \psi(t) | \vec{r} \rangle$,

$\langle \vec{r} | \widehat{O} | \psi(t) \rangle = \widehat{O} (\vec{r}, -i \hbar \vec{\grad}) \psi(\vec{r}, t), \langle \vec{O} \rangle = \int d^3 \vec{r} \psi(\vec{r}, t)^* \vec{O}(\vec{r}, -i \hbar \vec{\grad}) \psi(\vec{r}, t)$

\textbf{Operators and wavefunction in momentum representation}

\subsection{\underline{Commuting operators}}


\subsection{\underline{Non-commuting operators and the general uncertainty principle}}

\subsection{\underline{Time evolution of expectation value of an operator and Ehrenfest's theorem}}

\subsection{\underline{The simple harmonic oscillator}}

\textbf{Raising and lowering operators}

\textbf{Normalized number state vectors}

\textbf{Wavefunctions in position representation}

\textbf{Classical simple harmonic oscillator}

\textbf{The quantum simple harmonic oscillator and coherent state}

%\subsection{\underline{3.1 Hilbert Space}}

%\subsection{\underline{3.2 Observables}}

%\subsection{\underline{3.3 Eigenfunctions of a Hermitian Operator}}

%\subsection{\underline{3.4 Generalized Statistical Interpretation}}

%\subsection{\underline{3.5 The Uncertainty Principle}}

%\subsection{\underline{3.6 Dirac Notation}}
