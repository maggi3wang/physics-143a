\section{5. MANY-PARTICLE SYSTEMS AND PERTURBATION THEORY} \hrule height 0.3pt \thinspace

\subsection{\underline{5.1 Identical particles}}

$\Psi(\vec{r}_1, \vec{r}_2, t)$, $H = -\frac{\hbar^2}{2m_1} \grad_1^2 - \frac{\hbar^2}{2m_2} \grad_2^2 + V(r_1, r_2, t)$

$\widehat{H} = \widehat{H}(1, 2) = \widehat{H}(2, 1) = -\frac{\hbar^2}{2m_1} \grad_1^2 - \frac{\hbar^2}{2m_2} \grad_2^2 + \widehat{V}(q_1, q_2)$, where $q_i = \vec{r}_i, s_i$ with $\vec{r}_i$ is the spatial coordinate and $s_i$ denote spin coordinate.

P of finding particle 1 in volume $d^3 r_1$, etc.: $\int | \psi(r_1, r_2, t)|^2 d^3 r_1 d^3 r_2 = 1$

$\Psi(\vec{r}_1, \vec{r}_2, t) = \psi(r_1, r_2) e^{-iEt/\hbar}$, $-\frac{\hbar^2}{2m_1} \grad_1^2 \psi - \frac{\hbar}{2m_2} \grad_2^2 \psi + V \psi = E \psi$

\hrule height 0.07pt

Exchange operator $\widehat{P}_{\textrm{ex}}: 1 \leftrightarrow 2$, which exchanges the two particles.

$\widehat{P}_{\textrm{ex}} \Psi(q_1, q_2) = \Psi(q_2, q_1)$ and $\widehat{P}^2_{\textrm{ex}} \Psi(q_1, q_2) = \Psi(q_1, q_2)$

$\widehat{P}_{\textrm{ex}}$ has two eigenvalues $p_{\textrm{ex}} = \pm 1$

$[\widehat{P}_{\textrm{ex}}, \widehat{H}] = 0$. Can construct simulataenous eigenstates of $\widehat{P}_{\textrm{ex}}$ and $\widehat{H}(1, 2)$:

$\widehat{H} \Psi_{\pm}(q_1, q_2) = E \Psi_{\pm}(q_1, q_2)$, $\widehat{P}_\textrm{ex} \Psi_{\pm} (q_1, q_2) = \pm \Psi_{\pm}(q_1, q_2)$

\hrule height 0.07pt

Identical particles in QM come in two and only two classes:

1. Bosons: $\Psi_{+}(q_2, q_1) = \widehat{P}_{\textrm{ex}} \Psi_{+}(q_1, q_2) = + \Psi_{+}(q_1, q_2)$, $s=0,1,2,...$

2. Fermions: $\Psi_{-}(q_2, q_1) = \widehat{P}_{\textrm{ex}} \Psi_{-}(q_1, q_2) = - \Psi_{-}(q_1, q_2)$, $s=\frac{1}{2}, \frac{3}{2}, \frac{5}{2}$

\subsection{\underline{5.2 Identical noninteracting particles}}

$\widehat{H}(1, 2) = \frac{\widehat{\vec{p}}_1^2}{2m} + \frac{\widehat{\vec{p}}_2^2}{2m} + \widehat{V}(\widehat{q}_1) + \widehat{V}(\widehat{q}_2) = \widehat{H}(1) + \widehat{H}(2)$

$\widehat{H}(1) \psi_a (q_1) = E_a \psi_a(q_1)$, $\widehat{H}(2) \psi_a (q_2) = E_a \psi_a (q_2)$

Same set of eigenst, eigenval, and quantum nums: $\{\psi_a(q)\}, \{E_a\}, \{a \}$

$\Psi_{-}(q_1, q_2) = \frac{1}{\sqrt{N!}} \det \dots =\frac{1}{\sqrt{2}} \det \begin{bmatrix} \psi_a(q_1) & \psi_b(q_1) \\ \psi_a(q_2) & \psi_b(q_2) \end{bmatrix}$, Slater det.

Antisymmetrical, for fermions. Bosons: flip all minus signs into plus signs.

Pauli exclusion principle: two identical fermions can't have same quantum nums (or can't occupy the same state). Two bosons can occupy the same state.

\hrule height 0.07pt

\subsubsection{Bosons tend to congregate and fermions tend to avoid each other}

Particle in state $\psi_a(x)$ and another in state $\psi_b(x)$. These two states are orthogonal and normalized.

If distinguishable, $\psi(x_1, x_2) = \psi_a(x_1) \psi_b(x_2)$

If identical bosons, $\psi_{+} (x_1, x_2) = \frac{1}{\sqrt{2}} [\psi_a(x_1) \psi)b(x_2) + \psi_b(x_1) \psi_a(x_2)$

If identical fermions, $\psi_{-}(x_1, x_2) = \frac{1}{\sqrt{2}} [\psi_a(x_1) \psi_b(x_2) - \psi_b(x_1) \psi_a(x_2)]$

Separation of the two particles: $\langle (\Delta x)^2 \rangle = \langle (x_1 - x_2)^2 \rangle = \langle x_1^2 \rangle + \langle x_2^2 \rangle - 2 \langle x_1 x_2 \rangle$

1. Distinguishable particles

$\langle x_1^2 \rangle_{\textrm{dist}} = \int x_1^2 | \psi_a(x_1)|^2 dx_1 \int | \psi_b(x_2)|^2 dx_2 = \int x_1^2 | \psi_a (x_1)|^2 dx_1 = \langle x^2 \rangle_a$. Similarly, $\langle x_2^2 \rangle_{\textrm{dist}} = \langle x^2 \rangle_b$, $\langle x_1 x_2 \rangle_{\textrm{dist}} = \int x_1 | \psi_a(x_1)|^2 dx_1 \int x_2 | \psi_b (x_2)|^2 dx_2 = \langle x \rangle_a \langle x \rangle_b$

$\langle (\Delta x)^2 \rangle_{\textrm{dist}} = \langle x^2 \rangle_a + \langle x^2 \rangle_b - 2 \langle x \rangle_a \langle x \rangle_b$

2. Identical particles

$|\Psi_{\pm}(x_1, x_2)|^2 = \frac{1}{2}(|\psi_a(x_1)|^2 |\psi_b(x_2)|^2 + |\psi_b(x_1)|^2 |\psi_a(x_2)|^2 \pm \psi^{*}_a(x_1) \psi_b(x_1) \psi_b^*(x_2) \psi_a(x_2) \pm \psi_b^*(x_1)\psi_a(x_1) \psi_a^*(x_2) \psi_b(x_2))$

$\langle x_1^2 \rangle_{\pm} = \langle x_2^2 \rangle_{\pm} = \frac{1}{2} (\langle x^2 \rangle_a + \langle x^2 \rangle_b)$, $\langle x_1, x_2 \rangle = \langle x \rangle_a \langle x \rangle_b \pm |\langle x \rangle_{ab}|^2$

$\langle (x_1 - x_2)^2 \rangle_{\pm} = \langle x^2 \rangle_a + \langle x^2 \rangle_b - 2 \langle x \rangle_a \langle x \rangle_b \mp 2|\langle x \rangle_{ab}|^2$, $\langle (\Delta x)^2 \rangle \pm = \langle (\Delta x)^2 \rangle_{\textrm{dist}} \mp 2 | \langle x \rangle_{ab}|^2$

Id. bosons: spatially closer, id. fermions: apart, compared to distinguishable.

Purely QM effect that follows from sym. or antisym. of the wavefunction.

\hrule height 0.07pt

\subsubsection{$H_2$ molecule and covalent bond}
Two H atoms each in ground state and spatially far apart.

$\Psi_{\textrm{tot}-}(q_1, q_2) = \Psi(\vec{r}_1, \vec{r}_2) \chi(1, 2)$, where $\Psi(\vec{r}_1, \vec{r}_2)$ is the spatial part of the wavefn and $\chi(1, 2)$ is the spin part.

$\Psi_{\textrm{tot}-} = \Psi(\vec{r}_1, \vec{r}_2)_{+} \chi(1, 2)_{-}$, sym, produces a covalent bond.

$\Psi_{\textrm{tot}-} = \Psi(\vec{r}_1, \vec{r}_2)_{-} \chi(1, 2)_{+}$, antisym, electrons avoid each other spatially.

$\Psi_{\textrm{tot}-}(q_1, q_2) = \frac{1}{\sqrt{2}}(\psi_{100}(\vec{r}_1 - \vec{r}_0) \psi_{100}(\vec{r}_2 + \vec{r}_0) + \psi_{100}(\vec{r}_1 + \vec{r}_0) \psi_{100}(\vec{r}_2 - \vec{r}_0)) = \frac{1}{\sqrt{2}} (|\frac{1}{2} \frac{1}{2} (1) \rangle | \frac{1}{2} -\frac{1}{2} (2) \rangle - | \frac{1}{2} -\frac{1}{2}(1) \rangle | \frac{1}{2} \frac{1}{2} (2) \rangle)$

\hrule height 0.07pt

$d$-fold degen., energy level occupied by $N > 2d$ num of spin-half id fermions $\rightarrow$ color.

\hrule height 0.07pt

\subsection{\underline{5.3 Perturbation theory}}

Time-dependent Hamiltonian $\widehat{H}_0$ with known wavefunctions $|\psi_a^{(0)}\rangle$ and energies $E_a^{(0)}$, $\widehat{H}_0 |\phi_a^{(0)} \rangle = E_a^{(0)} | \psi_a^{(0)} \rangle$

SE w new Hamiltonian: $i \hbar \pdv{}{t} | \psi_n \rangle = (\widehat{H}_0 + \widehat{H}'(t) | \psi_n \rangle$. We call $\widehat{H}_0$ the unperturbed Hamiltonian and $\widehat{H}'(t)$ the perturbation, which could be time-dep.

\hrule height 0.07pt

\subsubsection{Time-independent perturbation theory}

$\widehat{H}'(t) = \widehat{H}'$. $\widehat{H} = \widehat{H}_0 + \widehat{H}'$ is TI. $\widehat{H} |\psi_n \rangle = E_n | \psi_n \rangle$, $(E_n - E_a^{(0)}) \langle \psi_a^{(0)}) \langle \psi_a^{(0)} | \psi_n \rangle = \sum_b H'_{ab} \langle \psi_b^{(0)} | \psi_n \rangle$, $H'_{ab} = \langle \psi_a^{(0)} | \widehat{H}' | \psi_n^{(0)} \rangle$: matrix element of perturbation in the unpert. states.

$(E_n^{(0)} - E_a^{(0)} + E_n^{(1)} + E_n^{(2)} + \dots) \langle \phi_a^{(0)} (|\psi_n^{(0)} \rangle + | \psi_n^{(1)}\rangle + | \psi_n^{(2)} \rangle + \dots) = \sum_b H'_{ab} \langle \phi_b^{(0)} (| \phi_n^{(0)} \rangle + | \psi_n^{(1)} \rangle + |\psi_n^{(2)} \rangle + \dots)$. SE in a diff form, all exact.

Now suppose $\widehat{H}'$ is small compared to $\widehat{H}_0$. The additional terms should be small. Perturbation theory involves solving above by organizing the corrections s.t. $E_n^{(2)}$ is smaller than $E_n^{(1)}$, $|\psi_n^{(2)} \rangle$  is smaller than $|\psi_n^{(1)}\rangle$, and so on.

\hrule height 0.07pt

\subsubsection{Nondegenerate time-independent perturbation theory}

Nondegenerate: any two unperturbed states $|\psi_a^{(0)}\rangle$ and $|\psi_b^{(0)} \rangle$ with $a \neq b$ have $E_a^{(0)} \neq E_b^{(0)}$

\textbf{Zeroth order}

$(E_n^{(0)} - E_a^{(0)}) \langle \psi_a^{(0)} | \psi_n^{(0)} \rangle = 0$.
$E_n = E_n^{(0)}$, $|\psi_n \rangle = |\psi_n^{(0)} \rangle$, no corrections.

\textbf{First order}

$(E_n^{(0)} - E_a^{(0)}) \langle \psi_a^{(0)} | \psi_n^{(1)} \rangle + E_n^{(1)} \langle \psi_a^{(0)} | \psi_n^{(0)} \rangle = \sum_b H'_{ab} \langle \psi_b^{(0)} | \psi_n^{(0)} \rangle$

$a = n$, $E_n^{(0)} - E_a^{(0)} = 0, \delta_{an} = 1$.
$a \neq n$, $E_n^{(0)} - E_a^{(0)} \neq 0$, $\delta_{an} = 0$.

$E_n = E_n^{(0)} + H'_{nm}$, $|\psi_n \rangle = | \psi_n^{(0)} - \sum_{m \neq n} \frac{H'_{mn}}{E^{(0)}_m - E^{(0)}_n} | \psi_m^{(0)} \rangle$

$|\frac{H'_{mn}}{E_m^{(0)} - E_n^{(0)}}| << 1$, matrix elements of the perturbation btwn the unperturbed states must be much smaller than the diff btwn corresponding unpert. E's. 

\textbf{Second order}:
$E_n = E_n^{(0)} + H'_{nn} - \sum_{m \neq n} \frac{|H'_{mn}|^2}{E_m^{(0)} - E_n^{(0)}}$

States of lower energy make pos contribution while states of higher energy make neg contribution.

\subsubsection{Degenerate time-independent perturbation theory}

Ex: unperturbed hydrogen atom where $|\psi^{(0)}_{nlm}\rangle$ w the same $n$ but diff $l$'s and $m$'s are degenerate.

Consider an unperturbed energy level that is $d$-fold degenerate w $d$ states $|\psi_n^{(0)}\rangle, |\psi_{n'}^{(0)}\rangle, \dots$, having the same energy $E_n^{(0)} = E_{n'}^{(0)} = \dots$

%$|\psi^{(0)} \rangle = \sum_{n'} c^{(0)}_{n'} | \psi_{n'}^{(0)} \rangle$

$(E^{(0)} - E^{(0)}_a) \langle \psi^{(0)}_a | \psi^{(1)} \rangle + E^{(1)} \langle \psi_a^{(0)} | \psi^{(0)} \rangle = \sum_b H'_{ab} \langle \psi_b^{(0)} | \psi^{(0)} \rangle$

Secular equation: $\det |H'_{nn'} - E^{(1)} \delta_{n n'}| = 0$

$|\Psi_n^0 \rangle = \sum_{n'} c_{n'}^{(0)} | \psi_{n'}^{(0)} \rangle$

If matrix elements of the pert. Hamiltonian are diagonal, $H'_{nn'} = E_n^{(1)} \delta_{n'n}$, then $\exists$ no cross terms that mix diff states $\rightarrow E_n^{(1)} = H'_{nn}$.

\subsection{\underline{5.4 Fine structure of hydrogen atom}}

\subsubsection{Relativistic kinetic energy correction}

Relativistic energy of the electron: $E = mc^2 \sqrt{1 + \frac{\vec{p}^2}{m^2 c^2}}$

$E = mc^2(1 + \frac{1}{2} \frac{\vec{p}^2}{m^2 c^2} - \frac{1}{2}\frac{1}{2}\frac{1}{2} (\frac{\vec{p}^2}{m^2 c^2})^2 + \dots) = mc^2 + \frac{\vec{p}^2}{2m} - \frac{\vec{p}^4}{8m^3c^4} + \dots$

KE perturbation: $\widehat{H}_k = -\frac{\widehat{\vec{p}}^4}{8 m^3 c^4}$

\subsubsection{Spin-orbit correction}

$V(r) = -e \Phi(r)$, where $\Phi(r)$ is the corresponding electric potential.

Supposing that the electron sees magnetic field $\vec{B}'$, it has additional energy $\widehat{H}_{\textrm{SO}} = - \widehat{\vec{\mu}} \cdot \widehat{\vec{B}}'$, where $\widehat{\vec{\mu}} = g \frac{-e}{2mc} \widehat{\vec{S}} = - \frac{e}{mc} \widehat{\vec{S}}$ is the magnetic moment and $g=2$ is the gyromagnetic ratio of the electron.

Thomas precession: electron is rotating and accelerating around tne nucleus, and it is not an inertial frame. $\vec{B}'_{\perp} = \vec{B}'$, and $\vec{B}'_{\perp} = \vec{B} = \frac{1}{2} \frac{\vec{E} \cross \vec{v}}{c}$

$\widehat{H}_{\textrm{SO}} = \frac{1}{2m^2c^2 r} \dv{\widehat{V}}{r} \widehat{\vec{L}} \cdot \widehat{\vec{S}}$. For hydrogenic atoms, $\widehat{V} = -\frac{Ze^2}{4 \pi \epsilon_0 r}$ and $\dv{\widehat{V}}{r} = \frac{Ze^2}{4 \pi \epsilon_0 r^2}$
$\rightarrow \widehat{H}_{\textrm{SO}} = \frac{Ze^2}{8 \pi \epsilon_0 m^2 c^2 r^3} \widehat{\vec{L}} \cdot \widehat{\vec{S}}$

\subsubsection{Darwin correction}

For states with $l=0$, no orbital angular momentum, no spin-orbit interaction.
$\widehat{H}_D = \frac{\hbar^2 Z e^2}{8 m^2 c^2 \epsilon_0} \delta^3 (\vec{r})$

