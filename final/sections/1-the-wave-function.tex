\section{1. THE WAVE FUNCTION} \hrule height 0.3pt \thinspace

\subsection{\underline{1.1 The Schr\"{o}dinger Equation}}

%$i \hbar \pdv{\Psi}{t} = - \frac{\hbar^2}{2m} \pdv[2]{\Psi}{x} + V \Psi$
$i \hbar \pdv{\Psi(\vec{r}, t)}{t} = -\frac{\hbar^2}{2m} \vec{\grad}^2 \Psi(\vec{r}, t) + V(\vec{r}, t) \Psi(\vec{r}, t)$,
$\vec{\grad}^2 = \pdv[2]{}{x} + \pdv[2]{}{y} + \pdv[3]{}{z}$ \\

Solve for the particle's wave function $\Psi(x, t)$

$\hbar = \frac{h}{2\pi} = 1.054572 \times 10^{-34}$ Js \\

\smallskip \hrule height 0.3pt

\subsection{\underline{1.2 The Statistical Interpretation}}

$\int_{a}^{b} |\Psi(x, t)|^2 dx = \left\{ \textrm{P of finding the particle btwn $a$ and $b$, at $t$} \right\}$ \\

\subsection{\underline{1.3 Probability}}
Standard deviation: $\sigma = \sqrt{\langle j^2 \rangle - \langle j \rangle ^2}$ \\
Expectation value of $x$ given $\Psi$: $\langle x \rangle = \int x |\Psi|^2 dx$ \\
Probability current: $J(x, t)=\frac{i \hbar}{2m} (\Psi \pdv{\Psi^*}{x} - \Psi^* \pdv{\Psi}{x})$

\subsection{\underline{1.4 Normalization}}

$\int_{-\infty}^{+\infty} |\Psi (x, t)|^2 dx = 1$ \\

The Schr\"odinger equation produces unitary time evolution: \\
$\vec{J} = - \frac{i \hbar}{2m} (\psi(\vec(r), t)^* \vec{\grad} \psi(\vec{r}, t) - \psi(\vec{r}, t) \vec{\grad} \psi(\vec{r}, t)^*)$ \\
The probability density satisfies the continuity equation, $\pdv{}{t} \mathcal{P} + \vec{\grad} \cdot J = 0$ \\
Because the probability for finding the particle at infinity is 0 (otherwise non-normalizable), $\vec{J} = 0$ at infinity. \\
Therefore, $\dv{}{t} \int_{-\infty}^{\infty} \mathcal{P} d^3 \vec{r} = \dv{}{t} P = 0$, where $P$ is the total probability $\rightarrow$
the total probability is constant in time.

\subsection{\underline{1.5 Momentum}}
For a particle in state $\Psi$, the expectation value of $x$ and $p$ is
    $$\langle x \rangle = \int_{-\infty}^{+\infty} x |\Psi(x, t)|^2 dx \qquad \langle p \rangle = m \frac{d \langle x \rangle}{dt} = -i \hbar \int (\Psi^* \pdv{\Psi}{x}) dx$$

Expectation value of any quantity, $Q(x, p)$:
$\langle Q(x, p) \rangle = \int \Psi^* Q(x, \frac{\hbar}{i} \pdv{}{x}) \Psi dx$

Position and momentum operators: $\widehat{\vec{r}} = \vec{r}$, $\widehat{\vec{p}} = -i \hbar \vec{\grad}$

\subsection{\underline{1.6: The Uncertainty Principle}}
The wavelength of $\Psi$ is related to the momentum of the particle by the de Broglie formula:
    $p = \frac{h}{\lambda} = \frac{2 \pi \hbar}{\lambda}$

Heisenberg's uncertainty principle: $\sigma_x \sigma_p \geq \frac{\hbar}{2}$

Commutation relation btwn position and momentum:
$$\widehat{p}_x (\widehat{x} \psi(x, t)) = -i \hbar \pdv{}{x} [x \psi(x, t)] = -i \hbar \psi(x, t) - i \hbar x \pdv{}{x} \psi(x, t)$$

$$\widehat{x} (\widehat{p}_x \psi(x, t)) = x (-i \hbar \pdv{}{x} \psi(x, t))$$

$\widehat{x} \widehat{p}_x - \widehat{p}_x \widehat{x} = [\widehat{x}, \widehat{p}_x] = i \hbar$

$[\widehat{x}_i, \widehat{p}_j ] = i \hbar \delta_{ij}, [\widehat{x}_i, \widehat{x}_j] = [\widehat{p}_i, \widehat{p}_j] = 0$. $\delta_{ij} = 1$ for $i=j$, $\delta_{ij} = 0$ for $i \neq j$ \\

Given three operators $\widehat{A}, \widehat{B}, \widehat{C}$, we have $[\widehat{A}, \widehat{B}\widehat{C}] = [\widehat{A}, \widehat{B}] \widehat{C} + \widehat{B} [\widehat{A}, \widehat{C}]$.

%\subsection{\underline{Other: Blackbody Spectrum}}
%$E = hv = \hbar \omega$ \\
%The wave number $k$ is $k = 2 \pi / \lambda = \omega / c$ \\
%Only two spin states occur (quantum number $m$ is +1 or -1). \\

% $\rho(\omega) = \frac{\hbar \omega^3}{\pi^2 c^3 (e^{\hbar \omega / {k_b T}} - 1)}$ \\
%Wien displacement law: $\lambda_{\text{max}} = \frac{2.90 \times 10^{-3} \text{mK}}{T}$ \\

