\documentclass[8pt,landscape]{article}
\usepackage[dvipsnames]{xcolor}
\usepackage{soul}
\usepackage{multicol}
\usepackage{calc}
\usepackage{ifthen}
\usepackage[landscape]{geometry}
\usepackage{graphicx}
\usepackage{amsmath, amssymb, amsthm}
\usepackage{latexsym, marvosym}
\usepackage{pifont}
\usepackage{lscape}
\usepackage{graphicx}
\usepackage{array}
\usepackage{booktabs}
\usepackage[bottom]{footmisc}
\usepackage{tikz}
\usetikzlibrary{shapes}
\usepackage{pdfpages}
\usepackage{wrapfig}
\usepackage{enumitem}
\usepackage{physics}
 \usepackage{esint}
\setlist[description]{leftmargin=0pt}
\usepackage{xfrac}
\usepackage[pdftex,
            pdfauthor={Maggie Wang},
            pdftitle={143a Cheat Sheet},
            pdfsubject={Cheat sheet for Physics 143a},
            pdfkeywords={cheatsheet} {pdf} {cheat} {sheet}
            ]{hyperref}
\usepackage{relsize}
\usepackage{graphicx,wrapfig,lipsum}
\usepackage{rotating}
\usepackage{vwcol}  

% FONT
\renewcommand{\familydefault}{\sfdefault}

 \newcommand\independent{\protect\mathpalette{\protect\independenT}{\perp}}
    \def\independenT#1#2{\mathrel{\setbox0\hbox{$#1#2$}%
    \copy0\kern-\wd0\mkern4mu\box0}} 
            
\newcommand{\noin}{\noindent}    

\geometry{top=.19in,left=.2in,right=.2in,bottom=.5in}

% Turn on the style
\usepackage{fancyhdr}
\pagestyle{fancy}
% Clear the header and footer
\fancyhead{}
\fancyfoot{}
% Set the right side of the footer to be the page number
\fancyfoot[R]{\thepage}

\setcounter{secnumdepth}{0}

\setlength{\parindent}{0pt}
\setlength{\parskip}{0pt plus 0.5ex}

% -----------------------------------------------------------------------

\usepackage{titlesec}
\titlespacing{\section}{0pt}{\parskip}{-\parskip}
\titlespacing{\subsection}{0pt}{\parskip}{-\parskip}
\titlespacing{\subsubsection}{2pt}{\parskip}{-\parskip}

\titleformat{\section}
{\color{RubineRed}\normalfont\scriptsize\bfseries}
{\color{RubineRed}\thesection}{1em}{}

\titleformat{\subsection}
{\color{Mulberry}\normalfont\scriptsize\bfseries}
{\color{Mulberry}\thesection}{1em}{}

\titleformat{\subsubsection}[runin]
{\color{Blue}\normalfont\scriptsize\bfseries}
{\color{Blue}\thesection}{1em}{}
[\;]

% https://tex.stackexchange.com/questions/352956/how-to-highlight-text-with-an-arbitrary-color
\newcommand{\hlc}[2][yellow]{{%
    \colorlet{foo}{#1}%
    \sethlcolor{foo}\hl{#2}}%
}

\definecolor{yel}{rgb}{1,1,0.8}
\definecolor{orng}{rgb}{1,0.9,0.7}
\definecolor{grn}{rgb}{0.7,0.99,0.7}

\begin{document}

\raggedright
\scriptsize
\begin{multicols*}{3}

% multicols parameters
% These lengths are set only within the two main columns
%\setlength{\columnseprule}{0.25pt}
\setlength{\premulticols}{1pt}
\setlength{\postmulticols}{1pt}
\setlength{\multicolsep}{1pt}
\setlength{\columnsep}{2pt}

\setlength\abovedisplayskip{0pt}
\setlength\belowdisplayskip{0pt}

% \renewcommand{\vec}[1]{\mathbf{#1}}
\let\oldhat\hat
\renewcommand{\hat}[1]{\oldhat{\mathbf{#1}}}

\newenvironment{Figure}
  {\par\medskip\noindent\minipage{\linewidth}}
  {\endminipage\par\medskip}

\enlargethispage{\baselineskip}

%%%%%%%%%%%%%%%%%%%%%%%%%%%%%%%%%%%%
%%% TITLE
%%%%%%%%%%%%%%%%%%%%%%%%%%%%%%%%%%%%

%%%%%%%%%%%%%%%%%%%%%%%%%%%%%%%%%%%%
%%% BEGIN CHEATSHEET
%%%%%%%%%%%%%%%%%%%%%%%%%%%%%%%%%%%%

\section{1. The Wave Function} \hrule height 0.3pt \thinspace

\subsection{\underline{1.1 The Schr\"{o}dinger Equation}}

%$i \hbar \pdv{\Psi}{t} = - \frac{\hbar^2}{2m} \pdv[2]{\Psi}{x} + V \Psi$
$i \hbar \pdv{\Psi(\vec{r}, t)}{t} = -\frac{\hbar^2}{2m} \vec{\grad}^2 \Psi(\vec{r}, t) + V(\vec{r}, t) \Psi(\vec{r}, t)$,
$\vec{\grad}^2 = \pdv[2]{}{x} + \pdv[2]{}{y} + \pdv[3]{}{z}$ \\

Solve for the particle's wave function $\Psi(x, t)$

$\hbar = \frac{h}{2\pi} = 1.054572 \times 10^{-34}$ Js \\

\subsection{\underline{1.2 The Statistical Interpretation}}

$\int_{a}^{b} |\Psi(x, t)|^2 dx = \left\{ \textrm{P of finding the particle btwn $a$ and $b$, at $t$} \right\}$ \\

\subsection{\underline{1.3 Probability}}
Standard deviation: $\sigma = \sqrt{\langle j^2 \rangle - \langle j \rangle ^2}$ \\
Expectation value of $x$ given $\Psi$: $\langle x \rangle = \int x |\Psi|^2 dx$ \\
Probability current: $J(x, t)=\frac{i \hbar}{2m} (\Psi \pdv{\Psi^*}{x} - \Psi^* \pdv{\Psi}{x})$

\subsection{\underline{1.4 Normalization}}

$\int_{-\infty}^{+\infty} |\Psi (x, t)|^2 dx = 1$ \\

The Schr\"odinger equation produces unitary time evolution: \\
$\vec{J} = - \frac{i \hbar}{2m} (\psi(\vec(r), t)^* \vec{\grad} \psi(\vec{r}, t) - \psi(\vec{r}, t) \vec{\grad} \psi(\vec{r}, t)^*)$ \\
The probability density satisfies the continuity equation, $\pdv{}{t} \mathcal{P} + \vec{\grad} \cdot J = 0$ \\
Because the probability for finding the particle at infinity is 0 (otherwise non-normalizable), $\vec{J} = 0$ at infinity. \\
Therefore, $\dv{}{t} \int_{-\infty}^{\infty} \mathcal{P} d^3 \vec{r} = \dv{}{t} P = 0$, where $P$ is the total probability $\rightarrow$
the total probability is constant in time.

\subsection{\underline{1.5 Momentum}}
For a particle in state $\Psi$, the expectation value of $x$ and $p$ is
    $$\langle x \rangle = \int_{-\infty}^{+\infty} x |\Psi(x, t)|^2 dx \qquad \langle p \rangle = m \frac{d \langle x \rangle}{dt} = -i \hbar \int (\Psi^* \pdv{\Psi}{x}) dx$$

To calculate the expectation value of any quantity, $Q(x, p)$:
$$\langle Q(x, p) \rangle = \int \Psi^* Q(x, \frac{\hbar}{i}, \pdv{}{x}) \Psi dx$$

Position and momentum operators: $\widehat{\vec{r}} = \vec{r}$, $\widehat{\vec{p}} = -i \hbar \vec{\grad}$

\subsection{\underline{1.6: The Uncertainty Principle}}
The wavelength of $\Psi$ is related to the momentum of the particle by the de Broglie formula:
    $p = \frac{h}{\lambda} = \frac{2 \pi \hbar}{\lambda}$

Heisenberg's uncertainty principle: $\sigma_x \sigma_p \geq \frac{\hbar}{2}$

Commutation relation btwn position and momentum:
$$\widehat{p}_x (\widehat{x} \psi(x, t)) = -i \hbar \pdv{}{x} [x \psi(x, t)] = -i \hbar \psi(x, t) - i \hbar x \pdv{}{x} \psi(x, t)$$
    $$\widehat{x} (\widehat{p}_x \psi(x, t)) = x (-i \hbar \pdv{}{x} \psi(x, t))$$

    $\widehat{x} \widehat{p}_x - \widehat{p}_x \widehat{x} = [\widehat{x}, \widehat{p}_x] = i \hbar$

    $[\widehat{x}_i, \widehat{p}_j ] = i \hbar \delta_{ij}, [\widehat{x}_i, \widehat{x}_j] = [\widehat{p}_i, \widehat{p}_j] = 0$, \\ where $\delta_{ij} = 1$ for $i=j$ and $\delta_{ij} = 0$ for $i \neq j$ \\

Given three operators $\widehat{A}, \widehat{B}, \widehat{C}$, we have $[\widehat{A}, \widehat{B}\widehat{C}] = [\widehat{A}, \widehat{B}] \widehat{C} + \widehat{B} [\widehat{A}, \widehat{C}]$.

%\subsection{\underline{Other: Blackbody Spectrum}}
%$E = hv = \hbar \omega$ \\
%The wave number $k$ is $k = 2 \pi / \lambda = \omega / c$ \\
%Only two spin states occur (quantum number $m$ is +1 or -1). \\

% $\rho(\omega) = \frac{\hbar \omega^3}{\pi^2 c^3 (e^{\hbar \omega / {k_b T}} - 1)}$ \\
%Wien displacement law: $\lambda_{\text{max}} = \frac{2.90 \times 10^{-3} \text{mK}}{T}$ \\


\section{2. Time-Independent Schr\"{o}dinger Equation} \hrule height 0.3pt \thinspace

\subsection{\underline{2.1 Stationary States}}
Separation of variables: $\Psi(x, t) = \psi(x) \varphi(t)$ \\

$\varphi(t) = e^{-iEt/\hbar}$, separable solutions: $\Psi(x, t) = \psi(x) e^{-iEt / \hbar}$ \\

Time-independent Schr\"{o}dinger equation:
$$-\frac{\hbar^2 }{2m} \dv[2]{\psi}{x} + V \psi = E \psi $$ \\

Why separable solutions? \\
1. Stationary states - time-dependence cancels out
    $$|\Psi(x,t)|^2 = \Psi^* \Psi = \psi^* e^{+iEt/\hbar} \psi e^{-iEt/\hbar} = |\Psi(x)|^2$$

    Same thing happens in calculating the expectation value of any dynamical variable. Every expectation value is constant in time. \\

2. There are states of definite total energy. The total energy (kinetic plus potential) is the Hamiltonian: $H(x, p) = \frac{p^2}{2m} + V(x)$. \\

Hamiltonian operator: $\widehat{H} = -\frac{\hbar^2}{2m} \pdv[2]{}{x} + V(x)$ \\
Thus the TISE can be written as $\widehat{H} \psi = E \psi$ \\
Variance of $H$: $\sigma_{H}^2 = \langle H^2 \rangle - \langle H \rangle ^2 = E^2 - E^2 = 0$ \\
A separable solution has the property that every measurement of the total energy is certain to return the value $E$. \\

3. The general solution is a linear combination of separable solutions. There is a different wave function for each allowed energy: $\Psi_1(x,t) = \psi_1(x) e^{-iE_1 t / \hbar}, \Psi_2(x,t) = \psi_2 (x) e^{-iE_2 t \hbar}, ...$

Now the time dependent Schr\"{o}dinger equation has the property that any linear combo of solutions is itself a solution.

$$\Psi(x, t) = \sum_{n=1}^{\infty} c_n \psi_n(x) e^{-iE_n t / \hbar} = \sum_{n=1}^{\infty} c_n \Psi_n(x, t)$$

Euler's formula: $e^{i \theta} = \cos \theta + i \sin \theta$

\subsection{\underline{2.2 The Infinite Square Well}}

Suppose
    $$V(x) = \begin{cases} 0, \textrm{if } 0 \leq x \leq a \\ \infty, \textrm{otherwise} \end{cases}$$

Classic simple harmonic oscillator, $\psi(x) = A \sin(kx) + B \cos(kx)$ \\

Boundary conditions \\

$$E_n = \frac{\hbar^2 k_n^2}{2m} - \frac{n^2 \pi^2 \hbar^2}{2ma^2}$$

$$\psi_n(x) = \sqrt{\frac{2}{a}} \sin(\frac{n \pi}{a} x)$$

$\psi_1$ is the ground state, others are excited states. \\

Properties of $\psi_n(x)$: \\
1. Alternatively even and odd. \\
2. As you go up in energy, each successive state has one more node. \\
3. They are mutually orthogonal, in the sense that $\int \psi_m(x)* \psi_n(x) dx = 0$ whenever $m \neq n$. \\

$\int \psi_m (x)* \psi_n(x) dx = \delta_{mn}$
where $\delta_{mn}$ (Kronecker delta) is 0 if $m \neq n$ and 1 if $m=n$. We say that the $\phi$'s are orthonormal.

4. They are complete, in the sense that any other function, $f(x)$, can be expressed as a linear combination of them (Fourier series), Dirichlet's theorem:
    $$f(x) = \sum_{n=1}^{\infty} c_n \psi_n(x) = \sqrt{\frac{2}{a}} \sum_{n=1}^{\infty} c_n \sin(\frac{n \pi}{a} x)$$

Fourier's trick: $c_n = \int \psi_n(x)^* f(x) dx$ \\
$$c_m = \frac{2}{a} \int_0^a f(x) \sin(\frac{m \pi x}{a}) dx$$

$|c_n|^2$ tells you the probability that a measurement of the energy would yield the value $E_n$. \\

Sum of these probabilities should be 1: 
    $$\sum_{n=1}^{\infty} |c_n|^2 = 1$$

The expectation value of the energy is
    $$\langle H \rangle = \sum_{n=1}^{\infty} |c_n|^2 E_n$$

Conservation of energy in QM

\subsection{\underline{2.3 The Harmonic Oscillator}}
Hooke's law: $F = -kx = m \frac{d^2 x}{d t^2}$ \\
Solution is $x(t) = A \sin(\omega t) + B \cos(\omega t)$, where $\omega = \sqrt{\frac{k}{m}}$, $V(x) = \frac{1}{2} k x^2$. \\

Taylor series: $V(x) = V(x_0) + V'(x_0) (x - x_0) + \frac{1}{2} V''(x_0)(x- x_0)^2 + \cdots$ \\

The Schr\"{o}diner Equation for the harmonic oscillator: $-\frac{\hbar^2}{2m} \dv[2]{\psi}{x} + \frac{1}{2} m \omega^2 x^2 \psi = E \psi$ \\
Introduce $\xi \equiv \sqrt{\frac{m \omega}{\hbar}} x$, so we have $\dv[2]{\psi}{\xi} = (\xi^2 - K) \psi$, where $K \equiv \frac{2E}{\hbar \omega}$. \\

The recursion formula: $a_{j+2} = \frac{(2j + 1 - K)}{(j + 1)(j + 2)} a_j$ \\
The complete solution is $h(\xi) = h_{\text{even}}(\xi) + h_{\text{odd}}(\xi)$ \\

$K = 2n + 1$, so $E_n = (n + \frac{1}{2}) \hbar \omega$ \\

Recursion formula for allowed $K$: $a_{j+2} = \frac{-2(n - j)}{(j+1)(j+2)} a_j$ \\

Hermite polynomials: $H_0 = 1$, $H_1 = 2 \xi$, $H_2 = 4 \xi^2 - 2$, $H_3 = 8 \xi^3 - 12 \xi$, $H_4 = 16 \xi^4 - 48 \xi^2 + 12$, $H_5 = 32 \xi^5 - 160 \xi^3 + 120 \xi$ \\

The normalized stationary states: $\psi_n(x) = (\frac{m \omega}{\pi \hbar})^{1/4} \frac{1}{\sqrt{2^n n!}} H_n(\xi) e^{-\xi^2 / 2}$ \\

Rodrigues formula: $H_n(\xi) = (-1)^n e^(\xi^2) (\dv{}{\xi})^n e^{-\xi^2}$

\subsection{\underline{2.4 The Free Particle}}
$$\pdv[2]{\xi}{x} = -k^2 \xi, k = \frac{\sqrt{2mE}}{\hbar}$$

General solution to the TISE: wave packet, $\Psi(x, t) = \frac{1}{\sqrt{2\pi}} \int_{-\infty}{\infty} \psi(k) e^{i (kx - \frac{\hbar k^2}{2m} t)} dk$ \\
$$\phi(x) = \frac{1}{\sqrt{2 \pi}} \int_{-\infty}^{+\infty} \Psi(x, 0) e^{-ikx} dx$$

Plancherel's theorem: $$f(x) = \frac{1}{\sqrt{2 \pi}} \int_{-\infty}^{\infty} F(k) e^{ikx} dk \leftrightarrow F(k) = \frac{1}{\sqrt{2\pi}} \int_{-\infty}^{\infty} f(x) e^{-kx} dx$$

$F(k)$ is the Fourier transform of $f(x)$; $f(x)$ is the inverse Fourier transform of $F(k)$ \\

Phase velocity: speed of individual ripples; grouop velocity: speed of the envelope \\

Dispersion relation: the formula for $\omega$ as a function of $k$

\subsection{\underline{2.5 The Delta-Function Potential}}

\subsection{\underline{2.6 The Finite Square Well}}



\section{3. Formalism} \hrule height 0.3pt \thinspace

\subsection{\underline{3.1 Hilbert Space}}

\subsection{\underline{3.2 Observables}}

\subsection{\underline{3.3 Eigenfunctions of a Hermitian Operator}}

\subsection{\underline{3.4 Generalized Statistical Interpretation}}

\subsection{\underline{3.5 The Uncertainty Principle}}

\subsection{\underline{3.6 Dirac Notation}}

\section{4. Three-dimensional systems} \hrule height 0.3pt \thinspace

\subsection{\underline{Three-dimensional infinite square well}}

\subsection{\underline{The Schr\"odinger equation in spherical coordinates}}

\subsection{\underline{Orbital angular momentum}}

\textbf{Spherical harmonics}




\section{5. MANY-PARTICLE SYSTEMS AND PERTURBATION THEORY} \hrule height 0.3pt \thinspace

\subsection{\underline{5.1 Identical particles}}

$\Psi(\vec{r}_1, \vec{r}_2, t)$, $H = -\frac{\hbar^2}{2m_1} \grad_1^2 - \frac{\hbar^2}{2m_2} \grad_2^2 + V(r_1, r_2, t)$

$\widehat{H} = \widehat{H}(1, 2) = \widehat{H}(2, 1) = -\frac{\hbar^2}{2m_1} \grad_1^2 - \frac{\hbar^2}{2m_2} \grad_2^2 + \widehat{V}(q_1, q_2)$, where $q_i = \vec{r}_i, s_i$ with $\vec{r}_i$ is the spatial coordinate and $s_i$ denote spin coordinate.

P of finding particle 1 in volume $d^3 r_1$, etc.: $\int | \psi(r_1, r_2, t)|^2 d^3 r_1 d^3 r_2 = 1$

$\Psi(\vec{r}_1, \vec{r}_2, t) = \psi(r_1, r_2) e^{-iEt/\hbar}$, $-\frac{\hbar^2}{2m_1} \grad_1^2 \psi - \frac{\hbar}{2m_2} \grad_2^2 \psi + V \psi = E \psi$

\hrule height 0.07pt

Exchange operator $\widehat{P}_{\textrm{ex}}: 1 \leftrightarrow 2$, which exchanges the two particles.

$\widehat{P}_{\textrm{ex}} \Psi(q_1, q_2) = \Psi(q_2, q_1)$ and $\widehat{P}^2_{\textrm{ex}} \Psi(q_1, q_2) = \Psi(q_1, q_2)$

$\widehat{P}_{\textrm{ex}}$ has two eigenvalues $p_{\textrm{ex}} = \pm 1$

$[\widehat{P}_{\textrm{ex}}, \widehat{H}] = 0$. Can construct simulataenous eigenstates of $\widehat{P}_{\textrm{ex}}$ and $\widehat{H}(1, 2)$:

$\widehat{H} \Psi_{\pm}(q_1, q_2) = E \Psi_{\pm}(q_1, q_2)$, $\widehat{P}_\textrm{ex} \Psi_{\pm} (q_1, q_2) = \pm \Psi_{\pm}(q_1, q_2)$

\hrule height 0.07pt

Identical particles in QM come in two and only two classes:

1. Bosons: $\Psi_{+}(q_2, q_1) = \widehat{P}_{\textrm{ex}} \Psi_{+}(q_1, q_2) = + \Psi_{+}(q_1, q_2)$, $s=0,1,2,...$

2. Fermions: $\Psi_{-}(q_2, q_1) = \widehat{P}_{\textrm{ex}} \Psi_{-}(q_1, q_2) = - \Psi_{-}(q_1, q_2)$, $s=\frac{1}{2}, \frac{3}{2}, \frac{5}{2}$

\subsection{\underline{5.2 Identical noninteracting particles}}

$\widehat{H}(1, 2) = \frac{\widehat{\vec{p}}_1^2}{2m} + \frac{\widehat{\vec{p}}_2^2}{2m} + \widehat{V}(\widehat{q}_1) + \widehat{V}(\widehat{q}_2) = \widehat{H}(1) + \widehat{H}(2)$

$\widehat{H}(1) \psi_a (q_1) = E_a \psi_a(q_1)$, $\widehat{H}(2) \psi_a (q_2) = E_a \psi_a (q_2)$

Same set of eigenst, eigenval, and quantum nums: $\{\psi_a(q)\}, \{E_a\}, \{a \}$

$\Psi_{-}(q_1, q_2) = \frac{1}{\sqrt{N!}} \det \dots =\frac{1}{\sqrt{2}} \det \begin{bmatrix} \psi_a(q_1) & \psi_b(q_1) \\ \psi_a(q_2) & \psi_b(q_2) \end{bmatrix}$, Slater det.

Antisymmetrical, for fermions. Bosons: flip all minus signs into plus signs.

Pauli exclusion principle: two identical fermions can't have same quantum nums (or can't occupy the same state). Two bosons can occupy the same state.

\hrule height 0.07pt

\subsubsection{Bosons tend to congregate and fermions tend to avoid each other}

Particle in state $\psi_a(x)$ and another in state $\psi_b(x)$. These two states are orthogonal and normalized.

If distinguishable, $\psi(x_1, x_2) = \psi_a(x_1) \psi_b(x_2)$

If identical bosons, $\psi_{+} (x_1, x_2) = \frac{1}{\sqrt{2}} [\psi_a(x_1) \psi)b(x_2) + \psi_b(x_1) \psi_a(x_2)$

If identical fermions, $\psi_{-}(x_1, x_2) = \frac{1}{\sqrt{2}} [\psi_a(x_1) \psi_b(x_2) - \psi_b(x_1) \psi_a(x_2)]$

Separation of the two particles: $\langle (\Delta x)^2 \rangle = \langle (x_1 - x_2)^2 \rangle = \langle x_1^2 \rangle + \langle x_2^2 \rangle - 2 \langle x_1 x_2 \rangle$

1. Distinguishable particles

$\langle x_1^2 \rangle_{\textrm{dist}} = \int x_1^2 | \psi_a(x_1)|^2 dx_1 \int | \psi_b(x_2)|^2 dx_2 = \int x_1^2 | \psi_a (x_1)|^2 dx_1 = \langle x^2 \rangle_a$. Similarly, $\langle x_2^2 \rangle_{\textrm{dist}} = \langle x^2 \rangle_b$, $\langle x_1 x_2 \rangle_{\textrm{dist}} = \int x_1 | \psi_a(x_1)|^2 dx_1 \int x_2 | \psi_b (x_2)|^2 dx_2 = \langle x \rangle_a \langle x \rangle_b$

$\langle (\Delta x)^2 \rangle_{\textrm{dist}} = \langle x^2 \rangle_a + \langle x^2 \rangle_b - 2 \langle x \rangle_a \langle x \rangle_b$

2. Identical particles

$|\Psi_{\pm}(x_1, x_2)|^2 = \frac{1}{2}(|\psi_a(x_1)|^2 |\psi_b(x_2)|^2 + |\psi_b(x_1)|^2 |\psi_a(x_2)|^2 \pm \psi^{*}_a(x_1) \psi_b(x_1) \psi_b^*(x_2) \psi_a(x_2) \pm \psi_b^*(x_1)\psi_a(x_1) \psi_a^*(x_2) \psi_b(x_2))$

$\langle x_1^2 \rangle_{\pm} = \langle x_2^2 \rangle_{\pm} = \frac{1}{2} (\langle x^2 \rangle_a + \langle x^2 \rangle_b)$, $\langle x_1, x_2 \rangle = \langle x \rangle_a \langle x \rangle_b \pm |\langle x \rangle_{ab}|^2$

$\langle (x_1 - x_2)^2 \rangle_{\pm} = \langle x^2 \rangle_a + \langle x^2 \rangle_b - 2 \langle x \rangle_a \langle x \rangle_b \mp 2|\langle x \rangle_{ab}|^2$, $\langle (\Delta x)^2 \rangle \pm = \langle (\Delta x)^2 \rangle_{\textrm{dist}} \mp 2 | \langle x \rangle_{ab}|^2$

Id. bosons: spatially closer, id. fermions: apart, compared to distinguishable.

Purely QM effect that follows from sym. or antisym. of the wavefunction.

\hrule height 0.07pt

\subsubsection{$H_2$ molecule and covalent bond}
Two H atoms each in ground state and spatially far apart.

$\Psi_{\textrm{tot}-}(q_1, q_2) = \Psi(\vec{r}_1, \vec{r}_2) \chi(1, 2)$, where $\Psi(\vec{r}_1, \vec{r}_2)$ is the spatial part of the wavefn and $\chi(1, 2)$ is the spin part.

$\Psi_{\textrm{tot}-} = \Psi(\vec{r}_1, \vec{r}_2)_{+} \chi(1, 2)_{-}$, sym, produces a covalent bond.

$\Psi_{\textrm{tot}-} = \Psi(\vec{r}_1, \vec{r}_2)_{-} \chi(1, 2)_{+}$, antisym, electrons avoid each other spatially.

$\Psi_{\textrm{tot}-}(q_1, q_2) = \frac{1}{\sqrt{2}}(\psi_{100}(\vec{r}_1 - \vec{r}_0) \psi_{100}(\vec{r}_2 + \vec{r}_0) + \psi_{100}(\vec{r}_1 + \vec{r}_0) \psi_{100}(\vec{r}_2 - \vec{r}_0)) = \frac{1}{\sqrt{2}} (|\frac{1}{2} \frac{1}{2} (1) \rangle | \frac{1}{2} -\frac{1}{2} (2) \rangle - | \frac{1}{2} -\frac{1}{2}(1) \rangle | \frac{1}{2} \frac{1}{2} (2) \rangle)$

\tiny
He: electrons are fermions, antisym under exch. $\psi_{100}(r, \theta, \phi) = \sqrt{\frac{Z^3}{\pi a^3}} e^{-\frac{Zr}{a}}$, $Z=2$, where $a$ is Bohr radius. Slater det, $\psi_{100}(\vec{r}_1) \psi_{100}(\vec{r}_2) \frac{1}{\sqrt{2}} (\chi_{\frac{1}{2} \frac{1}{2}} (1) \chi_{\frac{1}{2} -\frac{1}{2}} (2) - \chi_{\frac{1}{2} -\frac{1}{2}}(1) \chi_{\frac{1}{2} \frac{1}{2}} (2))$

\scriptsize

\hrule height 0.07pt

$d$-fold degen., E level occupied by $N > 2d$ \# of spin-half id fermions $\rightarrow$ color.

\hrule height 0.07pt

\subsection{\underline{5.3 Perturbation theory}}

Time-dependent Hamiltonian $\widehat{H}_0$ with known wavefunctions $|\psi_a^{(0)}\rangle$ and energies $E_a^{(0)}$, $\widehat{H}_0 |\psi_a^{(0)} \rangle = E_a^{(0)} | \psi_a^{(0)} \rangle$

SE w new Hamiltonian: $i \hbar \pdv{}{t} | \psi_n \rangle = (\widehat{H}_0 + \widehat{H}'(t) | \psi_n \rangle$. We call $\widehat{H}_0$ the unperturbed Hamiltonian and $\widehat{H}'(t)$ the perturbation, which could be time-dep.

\hrule height 0.07pt

\subsubsection{Time-independent perturbation theory}

$\widehat{H}'(t) = \widehat{H}'$. $\widehat{H} = \widehat{H}_0 + \widehat{H}'$ is TI. $\widehat{H} |\psi_n \rangle = E_n | \psi_n \rangle$, $(E_n - E_a^{(0)}) \langle \psi_a^{(0)}) \langle \psi_a^{(0)} | \psi_n \rangle = \sum_b H'_{ab} \langle \psi_b^{(0)} | \psi_n \rangle$, $H'_{ab} = \langle \psi_a^{(0)} | \widehat{H}' | \psi_n^{(0)} \rangle$: matrix element of perturbation in the unpert. states.

$(E_n^{(0)} - E_a^{(0)} + E_n^{(1)} + E_n^{(2)} + \dots) \langle \phi_a^{(0)} (|\psi_n^{(0)} \rangle + | \psi_n^{(1)}\rangle + | \psi_n^{(2)} \rangle + \dots) = \sum_b H'_{ab} \langle \phi_b^{(0)} (| \phi_n^{(0)} \rangle + | \psi_n^{(1)} \rangle + |\psi_n^{(2)} \rangle + \dots)$. SE in a diff form, all exact.

Now suppose $\widehat{H}'$ is small compared to $\widehat{H}_0$. The additional terms should be small. Perturbation theory involves solving above by organizing the corrections s.t. $E_n^{(2)}$ is smaller than $E_n^{(1)}$, $|\psi_n^{(2)} \rangle$  is smaller than $|\psi_n^{(1)}\rangle$, and so on.

\hrule height 0.07pt

\subsubsection{Nondegenerate time-independent perturbation theory}

Nondegenerate: any two unperturbed states $|\psi_a^{(0)}\rangle$ and $|\psi_b^{(0)} \rangle$ with $a \neq b$ have $E_a^{(0)} \neq E_b^{(0)}$

\textbf{Zeroth order}

$(E_n^{(0)} - E_a^{(0)}) \langle \psi_a^{(0)} | \psi_n^{(0)} \rangle = 0$.
$E_n = E_n^{(0)}$, $|\psi_n \rangle = |\psi_n^{(0)} \rangle$, no corrections.

\textbf{First order}

$(E_n^{(0)} - E_a^{(0)}) \langle \psi_a^{(0)} | \psi_n^{(1)} \rangle + E_n^{(1)} \langle \psi_a^{(0)} | \psi_n^{(0)} \rangle = \sum_b H'_{ab} \langle \psi_b^{(0)} | \psi_n^{(0)} \rangle$

$a = n$, $E_n^{(0)} - E_a^{(0)} = 0, \delta_{an} = 1$.
$a \neq n$, $E_n^{(0)} - E_a^{(0)} \neq 0$, $\delta_{an} = 0$.

$E_n = E_n^{(0)} + H'_{nm}$, $|\psi_n \rangle = | \psi_n^{(0)} - \sum_{m \neq n} \frac{H'_{mn}}{E^{(0)}_m - E^{(0)}_n} | \psi_m^{(0)} \rangle$

$|\frac{H'_{mn}}{E_m^{(0)} - E_n^{(0)}}| << 1$, $|\langle m | \widehat{H}' | n \rangle | << |E_n^{(0)} - E_m^{(0)}|$, $\alpha (\frac{\hbar}{m \omega})^2 (n+1)^2 << \hbar \omega$ matrix elements of the perturb. btwn the unpert. states must be much smaller than the diff btwn corresponding unpert. E's. 

\textbf{Second order}:
$E_n = E_n^{(0)} + H'_{nn} - \sum_{m \neq n} \frac{|H'_{mn}|^2}{E_m^{(0)} - E_n^{(0)}}$

States of lower energy make pos contribution while states of higher energy make neg contribution.

\subsubsection{Degenerate time-independent perturbation theory}

Ex: unperturbed hydrogen atom where $|\psi^{(0)}_{nlm}\rangle$ w the same $n$ but diff $l$'s and $m$'s are degenerate.

Consider an unperturbed energy level that is $d$-fold degenerate w $d$ states $|\psi_n^{(0)}\rangle, |\psi_{n'}^{(0)}\rangle, \dots$, having the same energy $E_n^{(0)} = E_{n'}^{(0)} = \dots$

%$|\psi^{(0)} \rangle = \sum_{n'} c^{(0)}_{n'} | \psi_{n'}^{(0)} \rangle$

$(E^{(0)} - E^{(0)}_a) \langle \psi^{(0)}_a | \psi^{(1)} \rangle + E^{(1)} \langle \psi_a^{(0)} | \psi^{(0)} \rangle = \sum_b H'_{ab} \langle \psi_b^{(0)} | \psi^{(0)} \rangle$

Secular equation: $\det |H'_{nn'} - E^{(1)} \delta_{n n'}| = 0$

$|\Psi_n^0 \rangle = \sum_{n'} c_{n'}^{(0)} | \psi_{n'}^{(0)} \rangle$

If matrix elements of the pert. Hamiltonian are diagonal, $H'_{nn'} = E_n^{(1)} \delta_{n'n}$, then $\exists$ no cross terms that mix diff states $\rightarrow E_n^{(1)} = H'_{nn}$.

\subsection{\underline{5.4 Fine structure of hydrogen atom}}

\subsubsection{Relativistic kinetic energy correction}

Relativistic energy of the electron: $E = mc^2 \sqrt{1 + \frac{\vec{p}^2}{m^2 c^2}}$

$E = mc^2(1 + \frac{1}{2} \frac{\vec{p}^2}{m^2 c^2} - \frac{1}{2}\frac{1}{2}\frac{1}{2} (\frac{\vec{p}^2}{m^2 c^2})^2 + \dots) = mc^2 + \frac{\vec{p}^2}{2m} - \frac{\vec{p}^4}{8m^3c^4} + \dots$

KE perturbation: $\widehat{H}_k = -\frac{\widehat{\vec{p}}^4}{8 m^3 c^4}$, all commute bc $|\vec{p}|$ is invar under rot.

Use $\widehat{H} = E_n = \frac{\widehat{p}^2}{2m} + \widehat{V}(r) \rightarrow \widehat{p}^2 = 2m(E_n - \widehat{V}(r))$

\subsubsection{Spin-orbit correction}

$V(r) = -e \Phi(r)$, where $\Phi(r)$ is the corresponding electric potential.

Supposing that the electron sees magnetic field $\vec{B}'$, it has additional energy $\widehat{H}_{\textrm{SO}} = - \widehat{\vec{\mu}} \cdot \widehat{\vec{B}}'$, where $\widehat{\vec{\mu}} = g \frac{-e}{2mc} \widehat{\vec{S}} = - \frac{e}{mc} \widehat{\vec{S}}$ is the magnetic moment and $g=2$ is the gyromagnetic ratio of the electron.

Thomas precession: electron is rotating and accelerating around tne nucleus, and it is not an inertial frame. $\vec{B}'_{\perp} = \vec{B}'$, and $\vec{B}'_{\perp} = \vec{B} = \frac{1}{2} \frac{\vec{E} \cross \vec{v}}{c}$

$\widehat{H}_{\textrm{SO}} = \frac{1}{2m^2c^2 r} \dv{\widehat{V}}{r} \widehat{\vec{L}} \cdot \widehat{\vec{S}}$. For hydrogenic atoms, $\widehat{V} = -\frac{Ze^2}{4 \pi \epsilon_0 r}$ and $\dv{\widehat{V}}{r} = \frac{Ze^2}{4 \pi \epsilon_0 r^2}$
$\rightarrow \widehat{H}_{\textrm{SO}} = \frac{Ze^2}{8 \pi \epsilon_0 m^2 c^2 r^3} \widehat{\vec{L}} \cdot \widehat{\vec{S}}$

Use $J=L+S$, $\widehat{\vec{L}} \cdot \widehat{\vec{S}} = \frac{1}{2} (\widehat{\vec{J}}^2 - \widehat{\vec{L}}^2 - \widehat{\vec{S}}^2)$. Doesn't commute w $L$, $S$.

$\langle n(l,s)j,m_j|..|n(l,s)j, m_j\rangle$

$\frac{1}{4} mc^2 Z^4 \alpha^4 \frac{1}{n^3 j(j+\frac{1}{2})}$ for $j=l+\frac{1}{2}$, $-\frac{1}{4} mc^2 Z^4 \alpha^4 \frac{1}{n^3 (j+\frac{1}{2}(j+1)}$ for $j=l - \frac{1}{2}$, $l\neq 0$

\subsubsection{Darwin correction}

For states with $l=0$, no orbital angular momentum, no spin-orbit interaction.
$\widehat{H}_D = \frac{\hbar^2 Z e^2}{8 m^2 c^2 \epsilon_0} \delta^3 (\vec{r})$

Invariant under rot and does not contain spin, commutes w all given operators.

$E_{n00}^{(1)} = \langle \psi_{n00} | \delta H_D | \psi_{n00} \rangle = \frac{\pi}{2} \frac{e^2 \hbar^2}{m^2 c^2} |\psi_{n00}(0)|^2 = \frac{1}{2} mc^2 Z^4 \alpha^4 \frac{1}{n^3}$

\end{multicols*}
\end{document}
